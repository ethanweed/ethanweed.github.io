\documentclass[]{article}
\usepackage{lmodern}
\usepackage{amssymb,amsmath}
\usepackage{ifxetex,ifluatex}
\usepackage{fixltx2e} % provides \textsubscript
\ifnum 0\ifxetex 1\fi\ifluatex 1\fi=0 % if pdftex
  \usepackage[T1]{fontenc}
  \usepackage[utf8]{inputenc}
\else % if luatex or xelatex
  \ifxetex
    \usepackage{mathspec}
  \else
    \usepackage{fontspec}
  \fi
  \defaultfontfeatures{Ligatures=TeX,Scale=MatchLowercase}
\fi
% use upquote if available, for straight quotes in verbatim environments
\IfFileExists{upquote.sty}{\usepackage{upquote}}{}
% use microtype if available
\IfFileExists{microtype.sty}{%
\usepackage{microtype}
\UseMicrotypeSet[protrusion]{basicmath} % disable protrusion for tt fonts
}{}
\usepackage[margin=1in]{geometry}
\usepackage{hyperref}
\hypersetup{unicode=true,
            pdfborder={0 0 0},
            breaklinks=true}
\urlstyle{same}  % don't use monospace font for urls
\usepackage{graphicx,grffile}
\makeatletter
\def\maxwidth{\ifdim\Gin@nat@width>\linewidth\linewidth\else\Gin@nat@width\fi}
\def\maxheight{\ifdim\Gin@nat@height>\textheight\textheight\else\Gin@nat@height\fi}
\makeatother
% Scale images if necessary, so that they will not overflow the page
% margins by default, and it is still possible to overwrite the defaults
% using explicit options in \includegraphics[width, height, ...]{}
\setkeys{Gin}{width=\maxwidth,height=\maxheight,keepaspectratio}
\IfFileExists{parskip.sty}{%
\usepackage{parskip}
}{% else
\setlength{\parindent}{0pt}
\setlength{\parskip}{6pt plus 2pt minus 1pt}
}
\setlength{\emergencystretch}{3em}  % prevent overfull lines
\providecommand{\tightlist}{%
  \setlength{\itemsep}{0pt}\setlength{\parskip}{0pt}}
\setcounter{secnumdepth}{0}
% Redefines (sub)paragraphs to behave more like sections
\ifx\paragraph\undefined\else
\let\oldparagraph\paragraph
\renewcommand{\paragraph}[1]{\oldparagraph{#1}\mbox{}}
\fi
\ifx\subparagraph\undefined\else
\let\oldsubparagraph\subparagraph
\renewcommand{\subparagraph}[1]{\oldsubparagraph{#1}\mbox{}}
\fi

%%% Use protect on footnotes to avoid problems with footnotes in titles
\let\rmarkdownfootnote\footnote%
\def\footnote{\protect\rmarkdownfootnote}

%%% Change title format to be more compact
\usepackage{titling}

% Create subtitle command for use in maketitle
\providecommand{\subtitle}[1]{
  \posttitle{
    \begin{center}\large#1\end{center}
    }
}

\setlength{\droptitle}{-2em}

  \title{}
    \pretitle{\vspace{\droptitle}}
  \posttitle{}
    \author{}
    \preauthor{}\postauthor{}
    \date{}
    \predate{}\postdate{}
  

\begin{document}

\section{Ethan Weed}\label{ethan-weed}

Associate Professor of Linguistics, Aarhus University, Aarhus, Denmark\\
Email: \href{mailto:ethan@cc.au.dk}{\nolinkurl{ethan@cc.au.dk}}

\subsection{Education and positions}\label{education-and-positions}

2015- present: Associate Professor, Linguistics, Aarhus University\\
2019: Visiting Researcher, Department of Psychology, University of
Connecticut, USA\\
2014: Visiting Researcher, Department of Psychology, University of
Connecticut, USA\\
2011-2014: Assistant Professor, Linguistics, Aarhus University\\
2011: Research Assistant at MindLab, Center for Functionally Integrative
Neuroscience (CFIN), Aarhus University Hospital\\
2011 (June 8): PhD in Linguistics from Aarhus University\\
2009-2011: PhD fellow Dept. Linguistics, Aarhus University,\\
2008: Research assistant at CFIN\\
2007: MA in Cognitive Semiotics, Aarhus University\\
2003: BA in Spanish from Aarhus University\\
1996: BA in Ecology from Hampshire College, USA

\subsection{Primary research
interests}\label{primary-research-interests}

Clinical Linguistics, Acoustic Phonetics, Clinical Pragmatics, Language
Acquisition in Special Populations, Electrophysiological Methods,
Natural Language Processing

\subsection{Teaching: Courses Taught}\label{teaching-courses-taught}

Child Language Acquisition: Corpus-Based Approaches; Cognitive
Neurolinguistics; Language of the Brain; Psycholinguistics of Second
Language Acquisition; Language Challenges in the Classroom; Statistics
for Linguists; Development of Language: Language Acquisition; Auditory
Neuroscience and Speech Processing; Research Workshop: Dyslexia;
Research Workshop: Auditory Processing; Child Language Acquisition;
Psycholinguistics; Humanities Elective Language and Cognition; Language,
Cognition, and the Brain; Phonetics and Phonology; Studium Generale
(Introduction to Philosophy of Science); Text Analysis

\subsection{Dissemination}\label{dissemination}

Over 20 invited talks at e.g.~DuPont Nutrition and Health, Novozymes
A/S, the Danish Organization of Speech Language Pathologists (FTHF), and
Chinese Academy of Sciences

\subsection{Research Grants}\label{research-grants}

2019-2024: Co-Investigator, National Institute on Deafness and Other
Communication Disorders: ``Early Predictors to School Age Language:
Individual and Interactional Child and Parent Factors'', Letitia Naigles
(PI)\\
2018-2019: AU Funding for Language as a Tool for Learning Center
2013-2018: Interacting Minds small grants for projects: ``Music, Voice,
and Auditory Brainstem Response'' ``Hearing What You're Thinking''
``Clinical Voices'' ``How well do cloze tests really test reading
comprehension in Danish higher-education students?''

\paragraph{Reviews for International
Journals}\label{reviews-for-international-journals}

Aphasiology; Dyslexia; Journal of Child Language; Journal of
Neuroscience; Proc. Nat. Acad. Sci (PNAS); Psychology of Language and
Communication; Journal of Neurolinguistics


\end{document}
