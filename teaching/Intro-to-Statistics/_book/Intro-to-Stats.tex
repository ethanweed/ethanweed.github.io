\documentclass[]{article}
\usepackage{lmodern}
\usepackage{amssymb,amsmath}
\usepackage{ifxetex,ifluatex}
\usepackage{fixltx2e} % provides \textsubscript
\ifnum 0\ifxetex 1\fi\ifluatex 1\fi=0 % if pdftex
  \usepackage[T1]{fontenc}
  \usepackage[utf8]{inputenc}
\else % if luatex or xelatex
  \ifxetex
    \usepackage{mathspec}
  \else
    \usepackage{fontspec}
  \fi
  \defaultfontfeatures{Ligatures=TeX,Scale=MatchLowercase}
\fi
% use upquote if available, for straight quotes in verbatim environments
\IfFileExists{upquote.sty}{\usepackage{upquote}}{}
% use microtype if available
\IfFileExists{microtype.sty}{%
\usepackage[]{microtype}
\UseMicrotypeSet[protrusion]{basicmath} % disable protrusion for tt fonts
}{}
\PassOptionsToPackage{hyphens}{url} % url is loaded by hyperref
\usepackage[unicode=true]{hyperref}
\hypersetup{
            pdftitle={A Very Gentle Introduction to Statistics},
            pdfauthor={Ethan Weed},
            pdfborder={0 0 0},
            breaklinks=true}
\urlstyle{same}  % don't use monospace font for urls
\usepackage[margin=1in]{geometry}
\usepackage{natbib}
\bibliographystyle{plainnat}
\usepackage{color}
\usepackage{fancyvrb}
\newcommand{\VerbBar}{|}
\newcommand{\VERB}{\Verb[commandchars=\\\{\}]}
\DefineVerbatimEnvironment{Highlighting}{Verbatim}{commandchars=\\\{\}}
% Add ',fontsize=\small' for more characters per line
\usepackage{framed}
\definecolor{shadecolor}{RGB}{248,248,248}
\newenvironment{Shaded}{\begin{snugshade}}{\end{snugshade}}
\newcommand{\KeywordTok}[1]{\textcolor[rgb]{0.13,0.29,0.53}{\textbf{#1}}}
\newcommand{\DataTypeTok}[1]{\textcolor[rgb]{0.13,0.29,0.53}{#1}}
\newcommand{\DecValTok}[1]{\textcolor[rgb]{0.00,0.00,0.81}{#1}}
\newcommand{\BaseNTok}[1]{\textcolor[rgb]{0.00,0.00,0.81}{#1}}
\newcommand{\FloatTok}[1]{\textcolor[rgb]{0.00,0.00,0.81}{#1}}
\newcommand{\ConstantTok}[1]{\textcolor[rgb]{0.00,0.00,0.00}{#1}}
\newcommand{\CharTok}[1]{\textcolor[rgb]{0.31,0.60,0.02}{#1}}
\newcommand{\SpecialCharTok}[1]{\textcolor[rgb]{0.00,0.00,0.00}{#1}}
\newcommand{\StringTok}[1]{\textcolor[rgb]{0.31,0.60,0.02}{#1}}
\newcommand{\VerbatimStringTok}[1]{\textcolor[rgb]{0.31,0.60,0.02}{#1}}
\newcommand{\SpecialStringTok}[1]{\textcolor[rgb]{0.31,0.60,0.02}{#1}}
\newcommand{\ImportTok}[1]{#1}
\newcommand{\CommentTok}[1]{\textcolor[rgb]{0.56,0.35,0.01}{\textit{#1}}}
\newcommand{\DocumentationTok}[1]{\textcolor[rgb]{0.56,0.35,0.01}{\textbf{\textit{#1}}}}
\newcommand{\AnnotationTok}[1]{\textcolor[rgb]{0.56,0.35,0.01}{\textbf{\textit{#1}}}}
\newcommand{\CommentVarTok}[1]{\textcolor[rgb]{0.56,0.35,0.01}{\textbf{\textit{#1}}}}
\newcommand{\OtherTok}[1]{\textcolor[rgb]{0.56,0.35,0.01}{#1}}
\newcommand{\FunctionTok}[1]{\textcolor[rgb]{0.00,0.00,0.00}{#1}}
\newcommand{\VariableTok}[1]{\textcolor[rgb]{0.00,0.00,0.00}{#1}}
\newcommand{\ControlFlowTok}[1]{\textcolor[rgb]{0.13,0.29,0.53}{\textbf{#1}}}
\newcommand{\OperatorTok}[1]{\textcolor[rgb]{0.81,0.36,0.00}{\textbf{#1}}}
\newcommand{\BuiltInTok}[1]{#1}
\newcommand{\ExtensionTok}[1]{#1}
\newcommand{\PreprocessorTok}[1]{\textcolor[rgb]{0.56,0.35,0.01}{\textit{#1}}}
\newcommand{\AttributeTok}[1]{\textcolor[rgb]{0.77,0.63,0.00}{#1}}
\newcommand{\RegionMarkerTok}[1]{#1}
\newcommand{\InformationTok}[1]{\textcolor[rgb]{0.56,0.35,0.01}{\textbf{\textit{#1}}}}
\newcommand{\WarningTok}[1]{\textcolor[rgb]{0.56,0.35,0.01}{\textbf{\textit{#1}}}}
\newcommand{\AlertTok}[1]{\textcolor[rgb]{0.94,0.16,0.16}{#1}}
\newcommand{\ErrorTok}[1]{\textcolor[rgb]{0.64,0.00,0.00}{\textbf{#1}}}
\newcommand{\NormalTok}[1]{#1}
\usepackage{longtable,booktabs}
% Fix footnotes in tables (requires footnote package)
\IfFileExists{footnote.sty}{\usepackage{footnote}\makesavenoteenv{long table}}{}
\usepackage{graphicx,grffile}
\makeatletter
\def\maxwidth{\ifdim\Gin@nat@width>\linewidth\linewidth\else\Gin@nat@width\fi}
\def\maxheight{\ifdim\Gin@nat@height>\textheight\textheight\else\Gin@nat@height\fi}
\makeatother
% Scale images if necessary, so that they will not overflow the page
% margins by default, and it is still possible to overwrite the defaults
% using explicit options in \includegraphics[width, height, ...]{}
\setkeys{Gin}{width=\maxwidth,height=\maxheight,keepaspectratio}
\IfFileExists{parskip.sty}{%
\usepackage{parskip}
}{% else
\setlength{\parindent}{0pt}
\setlength{\parskip}{6pt plus 2pt minus 1pt}
}
\setlength{\emergencystretch}{3em}  % prevent overfull lines
\providecommand{\tightlist}{%
  \setlength{\itemsep}{0pt}\setlength{\parskip}{0pt}}
\setcounter{secnumdepth}{5}
% Redefines (sub)paragraphs to behave more like sections
\ifx\paragraph\undefined\else
\let\oldparagraph\paragraph
\renewcommand{\paragraph}[1]{\oldparagraph{#1}\mbox{}}
\fi
\ifx\subparagraph\undefined\else
\let\oldsubparagraph\subparagraph
\renewcommand{\subparagraph}[1]{\oldsubparagraph{#1}\mbox{}}
\fi

% set default figure placement to htbp
\makeatletter
\def\fps@figure{htbp}
\makeatother

\usepackage{booktabs}
\usepackage{amsthm}
\makeatletter
\def\thm@space@setup{%
  \thm@preskip=8pt plus 2pt minus 4pt
  \thm@postskip=\thm@preskip
}
\makeatother

\title{A Very Gentle Introduction to Statistics}
\author{Ethan Weed}
\date{17 April, 2020}

\begin{document}
\maketitle

{
\setcounter{tocdepth}{2}
\tableofcontents
}
\section{Getting Data Into R.}\label{getting-data-into-r.}

One of the most useful variable types for data analysis in R is the
dataframe. A dataframe is a little bit like a spreadsheet, in that it
has rows and columns, but we interact with it a little differently than
we do with a spreadsheet like Excel. In Excel, you can just go in and
click on a cell and change whatever you want. You can add plots on top
of your data, and mess around however you please. This can be great for
taking a quick look at data, but it is no good for actual analysis,
becasue there is too much room for human error. It is too easy to
accidently change something and mess up all your data. Dataframes in R
are a little more structured than a spreadsheet, and because we interact
with them \emph{programatically}, by writing code, even if we mess
something up, we can always trace back our steps, and figure out what
went wrong.

The easiest way to get your data into R is by importing it from a .csv
file. Here we look at data from students carrying out a version of the
Stroop color-naming task. The command \texttt{sep\ =\ ","} tells R that
the data uses commas to separate the columns. You could use other
charaters to separate the columns. As an example, Excel sometimes uses
semicolons instead of commas. In this case, we can just write
\texttt{sep\ =\ ";"}. If you don't use the \texttt{sep\ =} command, R
will just assume it is a comma, so if your data is comma-separated you
can just leave this command off.

\begin{Shaded}
\begin{Highlighting}[]
\NormalTok{df <-}\StringTok{ }\KeywordTok{read.csv}\NormalTok{(}\StringTok{"/Users/ethan/Documents/GitHub/ethanweed.github.io/r-tutorials/data/Stroop-raw-over-the-years.csv"}\NormalTok{, }\DataTypeTok{sep =} \StringTok{","}\NormalTok{)}
\end{Highlighting}
\end{Shaded}

There are many rows of data, and maybe we only want to work with the
data from 2015. We can easily make a new dataframe, with only the 2015
data:

\begin{Shaded}
\begin{Highlighting}[]
\NormalTok{df_}\DecValTok{2015}\NormalTok{ <-}\StringTok{ }\KeywordTok{subset}\NormalTok{(df, Year }\OperatorTok{==}\StringTok{ }\DecValTok{2015}\NormalTok{)}
\NormalTok{df_}\DecValTok{2015}
\end{Highlighting}
\end{Shaded}

\begin{verbatim}
##     Reading_NoInt Naming_Int Naming_NoInt Reading_Int Year
## 91           4.16       6.76         4.45        4.65 2015
## 92           4.35       7.73         4.78        4.46 2015
## 93           3.60       7.00         4.00        3.50 2015
## 94           3.90       9.03         4.60        6.30 2015
## 95           4.22       9.98         6.83        6.24 2015
## 96           4.31       6.51         5.78        4.08 2015
## 97           4.58      11.57         6.22        4.49 2015
## 98           3.93       7.33         5.50        4.26 2015
## 99           4.48       9.78         5.04        5.33 2015
## 100          3.56       7.86         4.16        3.81 2015
## 101          3.10      10.49         5.34        3.62 2015
## 102          4.11       9.17         5.22        3.97 2015
## 103          3.17       7.51         3.69        3.26 2015
## 104          3.85       9.92         6.12        4.28 2015
\end{verbatim}

You can use the \texttt{\$} symbol to select a single column from the
dataframe, using the column header:

\begin{Shaded}
\begin{Highlighting}[]
\NormalTok{Reading_no_interference <-}\StringTok{ }\NormalTok{df_}\DecValTok{2015}\OperatorTok{$}\NormalTok{Reading_NoInt}
\NormalTok{Reading_no_interference}
\end{Highlighting}
\end{Shaded}

\begin{verbatim}
##  [1] 4.16 4.35 3.60 3.90 4.22 4.31 4.58 3.93 4.48 3.56 3.10 4.11 3.17 3.85
\end{verbatim}

\bibliography{refs.bib}

\end{document}
